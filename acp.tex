% Options for packages loaded elsewhere
\PassOptionsToPackage{unicode}{hyperref}
\PassOptionsToPackage{hyphens}{url}
%
\documentclass[
]{article}
\usepackage{amsmath,amssymb}
\usepackage{lmodern}
\usepackage{iftex}
\ifPDFTeX
  \usepackage[T1]{fontenc}
  \usepackage[utf8]{inputenc}
  \usepackage{textcomp} % provide euro and other symbols
\else % if luatex or xetex
  \usepackage{unicode-math}
  \defaultfontfeatures{Scale=MatchLowercase}
  \defaultfontfeatures[\rmfamily]{Ligatures=TeX,Scale=1}
\fi
% Use upquote if available, for straight quotes in verbatim environments
\IfFileExists{upquote.sty}{\usepackage{upquote}}{}
\IfFileExists{microtype.sty}{% use microtype if available
  \usepackage[]{microtype}
  \UseMicrotypeSet[protrusion]{basicmath} % disable protrusion for tt fonts
}{}
\makeatletter
\@ifundefined{KOMAClassName}{% if non-KOMA class
  \IfFileExists{parskip.sty}{%
    \usepackage{parskip}
  }{% else
    \setlength{\parindent}{0pt}
    \setlength{\parskip}{6pt plus 2pt minus 1pt}}
}{% if KOMA class
  \KOMAoptions{parskip=half}}
\makeatother
\usepackage{xcolor}
\usepackage[margin=1in]{geometry}
\usepackage{color}
\usepackage{fancyvrb}
\newcommand{\VerbBar}{|}
\newcommand{\VERB}{\Verb[commandchars=\\\{\}]}
\DefineVerbatimEnvironment{Highlighting}{Verbatim}{commandchars=\\\{\}}
% Add ',fontsize=\small' for more characters per line
\usepackage{framed}
\definecolor{shadecolor}{RGB}{248,248,248}
\newenvironment{Shaded}{\begin{snugshade}}{\end{snugshade}}
\newcommand{\AlertTok}[1]{\textcolor[rgb]{0.94,0.16,0.16}{#1}}
\newcommand{\AnnotationTok}[1]{\textcolor[rgb]{0.56,0.35,0.01}{\textbf{\textit{#1}}}}
\newcommand{\AttributeTok}[1]{\textcolor[rgb]{0.77,0.63,0.00}{#1}}
\newcommand{\BaseNTok}[1]{\textcolor[rgb]{0.00,0.00,0.81}{#1}}
\newcommand{\BuiltInTok}[1]{#1}
\newcommand{\CharTok}[1]{\textcolor[rgb]{0.31,0.60,0.02}{#1}}
\newcommand{\CommentTok}[1]{\textcolor[rgb]{0.56,0.35,0.01}{\textit{#1}}}
\newcommand{\CommentVarTok}[1]{\textcolor[rgb]{0.56,0.35,0.01}{\textbf{\textit{#1}}}}
\newcommand{\ConstantTok}[1]{\textcolor[rgb]{0.00,0.00,0.00}{#1}}
\newcommand{\ControlFlowTok}[1]{\textcolor[rgb]{0.13,0.29,0.53}{\textbf{#1}}}
\newcommand{\DataTypeTok}[1]{\textcolor[rgb]{0.13,0.29,0.53}{#1}}
\newcommand{\DecValTok}[1]{\textcolor[rgb]{0.00,0.00,0.81}{#1}}
\newcommand{\DocumentationTok}[1]{\textcolor[rgb]{0.56,0.35,0.01}{\textbf{\textit{#1}}}}
\newcommand{\ErrorTok}[1]{\textcolor[rgb]{0.64,0.00,0.00}{\textbf{#1}}}
\newcommand{\ExtensionTok}[1]{#1}
\newcommand{\FloatTok}[1]{\textcolor[rgb]{0.00,0.00,0.81}{#1}}
\newcommand{\FunctionTok}[1]{\textcolor[rgb]{0.00,0.00,0.00}{#1}}
\newcommand{\ImportTok}[1]{#1}
\newcommand{\InformationTok}[1]{\textcolor[rgb]{0.56,0.35,0.01}{\textbf{\textit{#1}}}}
\newcommand{\KeywordTok}[1]{\textcolor[rgb]{0.13,0.29,0.53}{\textbf{#1}}}
\newcommand{\NormalTok}[1]{#1}
\newcommand{\OperatorTok}[1]{\textcolor[rgb]{0.81,0.36,0.00}{\textbf{#1}}}
\newcommand{\OtherTok}[1]{\textcolor[rgb]{0.56,0.35,0.01}{#1}}
\newcommand{\PreprocessorTok}[1]{\textcolor[rgb]{0.56,0.35,0.01}{\textit{#1}}}
\newcommand{\RegionMarkerTok}[1]{#1}
\newcommand{\SpecialCharTok}[1]{\textcolor[rgb]{0.00,0.00,0.00}{#1}}
\newcommand{\SpecialStringTok}[1]{\textcolor[rgb]{0.31,0.60,0.02}{#1}}
\newcommand{\StringTok}[1]{\textcolor[rgb]{0.31,0.60,0.02}{#1}}
\newcommand{\VariableTok}[1]{\textcolor[rgb]{0.00,0.00,0.00}{#1}}
\newcommand{\VerbatimStringTok}[1]{\textcolor[rgb]{0.31,0.60,0.02}{#1}}
\newcommand{\WarningTok}[1]{\textcolor[rgb]{0.56,0.35,0.01}{\textbf{\textit{#1}}}}
\usepackage{graphicx}
\makeatletter
\def\maxwidth{\ifdim\Gin@nat@width>\linewidth\linewidth\else\Gin@nat@width\fi}
\def\maxheight{\ifdim\Gin@nat@height>\textheight\textheight\else\Gin@nat@height\fi}
\makeatother
% Scale images if necessary, so that they will not overflow the page
% margins by default, and it is still possible to overwrite the defaults
% using explicit options in \includegraphics[width, height, ...]{}
\setkeys{Gin}{width=\maxwidth,height=\maxheight,keepaspectratio}
% Set default figure placement to htbp
\makeatletter
\def\fps@figure{htbp}
\makeatother
\setlength{\emergencystretch}{3em} % prevent overfull lines
\providecommand{\tightlist}{%
  \setlength{\itemsep}{0pt}\setlength{\parskip}{0pt}}
\setcounter{secnumdepth}{-\maxdimen} % remove section numbering
\ifLuaTeX
  \usepackage{selnolig}  % disable illegal ligatures
\fi
\IfFileExists{bookmark.sty}{\usepackage{bookmark}}{\usepackage{hyperref}}
\IfFileExists{xurl.sty}{\usepackage{xurl}}{} % add URL line breaks if available
\urlstyle{same} % disable monospaced font for URLs
\hypersetup{
  pdftitle={Untitled},
  hidelinks,
  pdfcreator={LaTeX via pandoc}}

\title{Untitled}
\author{}
\date{\vspace{-2.5em}2023-02-16}

\begin{document}
\maketitle

\hypertarget{objectifs-de-lacp}{%
\section{Objectifs de l'ACP}\label{objectifs-de-lacp}}

L'ACP permet de décrire un jeu de données, de le résumer, d'en réduire
la dimensionnalité. L'ACP réalisée sur les individus du tableau de
données répond à différentes questions :

\begin{itemize}
\item
  Etude des individus (i.e.~des athlètes) : deux athlètes sont proches
  s'ils ont des résultats similaires. On s'intéresse à la variabilité
  entre individus. Y a-t-il des similarités entre les individus pour
  toutes les variables ? Peut-on établir des profils d'athlètes ?
  Peut-on opposer un groupe d'individus à un autre ?
\item
  Etude des variables (i.e.~des performances) : on étudie les liaisons
  linéaires entre les variables. Les objectifs sont de résumer la
  matrice des corrélations et de chercher des variables synthétiques:
  peut-on résumer les performances des athlètes par un petit nombre de
  variables ?
\item
  Lien entre les deux études : peut-on caractériser des groups
  d'individus par des variables ?
\end{itemize}

\hypertarget{importation-des-donnuxe9es}{%
\section{Importation des données}\label{importation-des-donnuxe9es}}

\begin{Shaded}
\begin{Highlighting}[]
\NormalTok{dat }\OtherTok{\textless{}{-}} \FunctionTok{read.csv}\NormalTok{(}\StringTok{"C:/Users/21650/Desktop/analyse de donnée/Depenses.csv"}\NormalTok{,}\AttributeTok{sep =} \StringTok{","}\NormalTok{,}\AttributeTok{header =} \ConstantTok{TRUE}\NormalTok{,}\AttributeTok{row.names=}\DecValTok{1}\NormalTok{)}
\NormalTok{dat}
\end{Highlighting}
\end{Shaded}

\begin{verbatim}
##       pvp agr  cmi  tra  log  edu  acs  aco  def  det div
## 1872 18.0 0.5  0.1  6.7  0.5  2.1  2.0  0.0 26.4 41.5 2.1
## 1880 14.1 0.8  0.1 15.3  1.9  3.7  0.5  0.0 29.8 31.3 2.5
## 1890 13.6 0.7  0.7  6.8  0.6  7.1  0.7  0.0 33.8 34.4 1.7
## 1900 14.3 1.7  1.7  6.9  1.2  7.4  0.8  0.0 37.7 26.2 2.2
## 1903 10.3 1.5  0.4  9.3  0.6  8.5  0.9  0.0 38.4 27.2 3.0
## 1906 13.4 1.4  0.5  8.1  0.7  8.6  1.8  0.0 38.5 25.3 1.9
## 1909 13.5 1.1  0.5  9.0  0.6  9.0  3.4  0.0 36.8 23.5 2.6
## 1912 12.9 1.4  0.3  9.4  0.6  9.3  4.3  0.0 41.1 19.4 1.3
## 1915 12.3 0.3  0.1 11.9  2.4  3.7  1.7  1.9 42.4 23.1 0.2
## 1923  7.6 1.2  3.2  5.1  0.6  5.6  1.8 10.0 29.0 35.0 0.9
## 1926 10.5 0.3  0.4  4.5  1.8  6.6  2.1 10.1 19.9 41.6 2.3
## 1929 10.0 0.6  0.6  9.0  1.0  8.1  3.2 11.8 28.0 25.8 2.0
## 1932 10.6 0.8  0.3  8.9  3.0 10.0  6.4 13.4 27.4 19.2 0.0
## 1935  8.8 2.6  1.4  7.8  1.4 12.4  6.2 11.3 29.3 18.5 0.4
## 1938 10.1 1.1  1.2  5.9  1.4  9.5  6.0  5.9 40.7 18.2 0.0
## 1947 15.6 1.6 10.0 11.4  7.6  8.8  4.8  3.4 32.2  4.6 0.0
## 1950 11.2 1.3 16.5 12.4 15.8  8.1  4.9  3.4 20.7  4.2 1.5
## 1953 12.9 1.5  7.0  7.9 12.1  8.1  5.3  3.9 36.1  5.2 0.0
## 1956 10.9 5.3  9.7  7.6  9.6  9.4  8.5  4.6 28.2  6.2 0.0
## 1959 13.1 4.4  7.3  5.7  9.8 12.5  8.0  5.0 26.7  7.5 0.0
## 1962 12.8 4.7  7.5  6.6  6.8 15.7  9.7  5.3 24.5  6.4 0.1
## 1965 12.4 4.3  8.4  9.1  6.0 19.5 10.6  4.7 19.8  3.5 1.8
## 1968 11.4 6.0  9.5  5.9  5.0 21.1 10.7  4.2 20.0  4.4 1.9
## 1971 12.8 2.8  7.1  8.5  4.0 23.8 11.3  3.7 18.8  7.2 0.0
\end{verbatim}

\hypertarget{acp-normuxe9e-avec-le-package-factominer-et-interpruxe9tation-de-lacp.}{%
\section{ACP normée avec le package FactoMineR et interprétation de
l'ACP.}\label{acp-normuxe9e-avec-le-package-factominer-et-interpruxe9tation-de-lacp.}}

Utilisons maintenant la fonction PCA pour retrouver les résultats
obtenus précédemment.

\begin{itemize}
\item
  On va ajouter deux types de variables comme variables supplémentaires
  : on ajoute les variables ``Rank'' and ``Points'' comme variables
  continues illustratives et la variable ``compétition'' comme variable
  qualitative illustrative. Les variables illustratives n'influencent
  pas la construction des composantes principales de l'analyse.
\item
  Notons que nous utilisons le package factoextra plutôt que FactoMineR
  pour la qualité de ces graphiques.
\end{itemize}

\begin{Shaded}
\begin{Highlighting}[]
\FunctionTok{library}\NormalTok{(FactoMineR)}


\FunctionTok{summary}\NormalTok{(dat)}
\end{Highlighting}
\end{Shaded}

\begin{verbatim}
##       pvp             agr             cmi              tra        
##  Min.   : 7.60   Min.   :0.300   Min.   : 0.100   Min.   : 4.500  
##  1st Qu.:10.57   1st Qu.:0.800   1st Qu.: 0.400   1st Qu.: 6.675  
##  Median :12.60   Median :1.400   Median : 1.300   Median : 8.000  
##  Mean   :12.21   Mean   :1.996   Mean   : 3.938   Mean   : 8.321  
##  3rd Qu.:13.43   3rd Qu.:2.650   3rd Qu.: 7.350   3rd Qu.: 9.150  
##  Max.   :18.00   Max.   :6.000   Max.   :16.500   Max.   :15.300  
##       log              edu              acs              aco        
##  Min.   : 0.500   Min.   : 2.100   Min.   : 0.500   Min.   : 0.000  
##  1st Qu.: 0.675   1st Qu.: 7.325   1st Qu.: 1.800   1st Qu.: 0.000  
##  Median : 1.850   Median : 8.700   Median : 4.550   Median : 3.800  
##  Mean   : 3.958   Mean   : 9.942   Mean   : 4.817   Mean   : 4.275  
##  3rd Qu.: 6.200   3rd Qu.:10.600   3rd Qu.: 6.800   3rd Qu.: 5.450  
##  Max.   :15.800   Max.   :23.800   Max.   :11.300   Max.   :13.400  
##       def             det             div       
##  Min.   :18.80   Min.   : 3.50   Min.   :0.000  
##  1st Qu.:25.93   1st Qu.: 6.35   1st Qu.:0.000  
##  Median :29.15   Median :19.30   Median :1.400  
##  Mean   :30.26   Mean   :19.14   Mean   :1.183  
##  3rd Qu.:37.02   3rd Qu.:26.45   3rd Qu.:2.025  
##  Max.   :42.40   Max.   :41.60   Max.   :3.000
\end{verbatim}

\begin{Shaded}
\begin{Highlighting}[]
\NormalTok{dat}
\end{Highlighting}
\end{Shaded}

\begin{verbatim}
##       pvp agr  cmi  tra  log  edu  acs  aco  def  det div
## 1872 18.0 0.5  0.1  6.7  0.5  2.1  2.0  0.0 26.4 41.5 2.1
## 1880 14.1 0.8  0.1 15.3  1.9  3.7  0.5  0.0 29.8 31.3 2.5
## 1890 13.6 0.7  0.7  6.8  0.6  7.1  0.7  0.0 33.8 34.4 1.7
## 1900 14.3 1.7  1.7  6.9  1.2  7.4  0.8  0.0 37.7 26.2 2.2
## 1903 10.3 1.5  0.4  9.3  0.6  8.5  0.9  0.0 38.4 27.2 3.0
## 1906 13.4 1.4  0.5  8.1  0.7  8.6  1.8  0.0 38.5 25.3 1.9
## 1909 13.5 1.1  0.5  9.0  0.6  9.0  3.4  0.0 36.8 23.5 2.6
## 1912 12.9 1.4  0.3  9.4  0.6  9.3  4.3  0.0 41.1 19.4 1.3
## 1915 12.3 0.3  0.1 11.9  2.4  3.7  1.7  1.9 42.4 23.1 0.2
## 1923  7.6 1.2  3.2  5.1  0.6  5.6  1.8 10.0 29.0 35.0 0.9
## 1926 10.5 0.3  0.4  4.5  1.8  6.6  2.1 10.1 19.9 41.6 2.3
## 1929 10.0 0.6  0.6  9.0  1.0  8.1  3.2 11.8 28.0 25.8 2.0
## 1932 10.6 0.8  0.3  8.9  3.0 10.0  6.4 13.4 27.4 19.2 0.0
## 1935  8.8 2.6  1.4  7.8  1.4 12.4  6.2 11.3 29.3 18.5 0.4
## 1938 10.1 1.1  1.2  5.9  1.4  9.5  6.0  5.9 40.7 18.2 0.0
## 1947 15.6 1.6 10.0 11.4  7.6  8.8  4.8  3.4 32.2  4.6 0.0
## 1950 11.2 1.3 16.5 12.4 15.8  8.1  4.9  3.4 20.7  4.2 1.5
## 1953 12.9 1.5  7.0  7.9 12.1  8.1  5.3  3.9 36.1  5.2 0.0
## 1956 10.9 5.3  9.7  7.6  9.6  9.4  8.5  4.6 28.2  6.2 0.0
## 1959 13.1 4.4  7.3  5.7  9.8 12.5  8.0  5.0 26.7  7.5 0.0
## 1962 12.8 4.7  7.5  6.6  6.8 15.7  9.7  5.3 24.5  6.4 0.1
## 1965 12.4 4.3  8.4  9.1  6.0 19.5 10.6  4.7 19.8  3.5 1.8
## 1968 11.4 6.0  9.5  5.9  5.0 21.1 10.7  4.2 20.0  4.4 1.9
## 1971 12.8 2.8  7.1  8.5  4.0 23.8 11.3  3.7 18.8  7.2 0.0
\end{verbatim}

\begin{Shaded}
\begin{Highlighting}[]
\NormalTok{res.pca}\OtherTok{=}\FunctionTok{PCA}\NormalTok{(dat,}\AttributeTok{ncp =} \DecValTok{5}\NormalTok{,}\AttributeTok{graph =}\NormalTok{ F)}
\FunctionTok{head}\NormalTok{(res.pca}\SpecialCharTok{$}\NormalTok{eig) }
\end{Highlighting}
\end{Shaded}

\begin{verbatim}
##        eigenvalue percentage of variance cumulative percentage of variance
## comp 1  4.9733656              45.212415                          45.21241
## comp 2  2.0499357              18.635779                          63.84819
## comp 3  1.2899237              11.726579                          75.57477
## comp 4  0.9933332               9.030302                          84.60507
## comp 5  0.7082871               6.438974                          91.04405
## comp 6  0.5586638               5.078762                          96.12281
\end{verbatim}

\begin{Shaded}
\begin{Highlighting}[]
\FunctionTok{library}\NormalTok{(ggplot2)}
\FunctionTok{library}\NormalTok{(factoextra)}
\end{Highlighting}
\end{Shaded}

\begin{verbatim}
## Welcome! Want to learn more? See two factoextra-related books at https://goo.gl/ve3WBa
\end{verbatim}

\begin{Shaded}
\begin{Highlighting}[]
\FunctionTok{fviz\_screeplot}\NormalTok{(res.pca, }\AttributeTok{ncp=}\DecValTok{10}\NormalTok{)}
\end{Highlighting}
\end{Shaded}

\includegraphics{acp_files/figure-latex/unnamed-chunk-4-1.pdf}

On devrait retenir les 2 premiers axes.

\hypertarget{choix-du-nombre-daxes-uxe0-retenir}{%
\subsection{2. Choix du nombre d'axes à
retenir}\label{choix-du-nombre-daxes-uxe0-retenir}}

Trois critères devront être utlisés : taux d'inertie cumulé, critère de
Kaiser et critère du coude.

L'objet \(\tt eig\) est une matrice à trois colonnes contenants
respectivement les valeurs propres de l'ACP, la proportion de variance
de chaque composante et les variance cumulées par les composantes
principales.

\begin{Shaded}
\begin{Highlighting}[]
\FunctionTok{head}\NormalTok{(res.pca}\SpecialCharTok{$}\NormalTok{eig)}
\end{Highlighting}
\end{Shaded}

\begin{verbatim}
##        eigenvalue percentage of variance cumulative percentage of variance
## comp 1  4.9733656              45.212415                          45.21241
## comp 2  2.0499357              18.635779                          63.84819
## comp 3  1.2899237              11.726579                          75.57477
## comp 4  0.9933332               9.030302                          84.60507
## comp 5  0.7082871               6.438974                          91.04405
## comp 6  0.5586638               5.078762                          96.12281
\end{verbatim}

\begin{Shaded}
\begin{Highlighting}[]
\FunctionTok{fviz\_screeplot}\NormalTok{(res.pca, }\AttributeTok{ncp=}\DecValTok{10}\NormalTok{)}
\end{Highlighting}
\end{Shaded}

\includegraphics{acp_files/figure-latex/unnamed-chunk-5-1.pdf}

\begin{enumerate}
\def\labelenumi{\alph{enumi})}
\item
  Critère de kaiser : on remarque qu'il y a 4 axes dont les valeurs
  propres sont supérieures à 1 donc on retient 4 axes d'après ce
  critère.
\item
  Critère du taux d'inertie cumulée : On remarque que le taux d'inertie
  cumulé des 2 premiers axes est de 50.09\% qui est un taux important
  compte tenu du fait que nous avons 10 variables : on va donc, d'après
  ce critère, retenir les 2 premiers axes. (Remarquons que retenir 3
  axes pour u taux d'inertie de 64.13\% est envisageable aussi).
\item
  Critère du coude : On remarque que le coude se trouve au niveau du
  deuxième axe (voir la figure Scree plot), d'après ce critère, on
  devrait retenir les 2 premiers axes.
\end{enumerate}

En faisant une sorte de vote des 3 critères on devrait retenir les 2
premiers axes.

\hypertarget{interpruxe9tation-de-la-carte-des-variables}{%
\section{Interprétation de la carte des
variables}\label{interpruxe9tation-de-la-carte-des-variables}}

\begin{Shaded}
\begin{Highlighting}[]
\FunctionTok{names}\NormalTok{(res.pca}\SpecialCharTok{$}\NormalTok{var)}
\end{Highlighting}
\end{Shaded}

\begin{verbatim}
## [1] "coord"   "cor"     "cos2"    "contrib"
\end{verbatim}

\begin{Shaded}
\begin{Highlighting}[]
\NormalTok{res.pca}\SpecialCharTok{$}\NormalTok{var}\SpecialCharTok{$}\NormalTok{coord}
\end{Highlighting}
\end{Shaded}

\begin{verbatim}
##          Dim.1        Dim.2      Dim.3      Dim.4       Dim.5
## pvp -0.1735776  0.739763078  0.3414692 -0.1075577 -0.33993173
## agr  0.8185167  0.005917819  0.3664333 -0.1536319  0.03264335
## cmi  0.8335942  0.340221611 -0.1408087  0.2580179 -0.15124675
## tra -0.1371658  0.630558704 -0.3760145  0.2811104  0.54685001
## log  0.7216129  0.397716225 -0.3851610  0.2079565 -0.26297993
## edu  0.7868293 -0.136503738  0.4246375 -0.1158897  0.31781458
## acs  0.9332593 -0.100484313  0.1661336 -0.1507405  0.10433499
## aco  0.2889615 -0.807563082 -0.3745040  0.2021535 -0.01160538
## def -0.6123426  0.216145915 -0.2599027 -0.6367657  0.15400108
## det -0.8887490 -0.301471010  0.1607997  0.1793104 -0.17586311
## div -0.5481219  0.112364351  0.5365720  0.5046560  0.18416903
\end{verbatim}

\begin{Shaded}
\begin{Highlighting}[]
\NormalTok{  res.pca}\SpecialCharTok{$}\NormalTok{var}\SpecialCharTok{$}\NormalTok{cos2 }
\end{Highlighting}
\end{Shaded}

\begin{verbatim}
##          Dim.1        Dim.2      Dim.3      Dim.4        Dim.5
## pvp 0.03012919 5.472494e-01 0.11660125 0.01156866 0.1155535811
## agr 0.66996960 3.502058e-05 0.13427338 0.02360275 0.0010655882
## cmi 0.69487928 1.157507e-01 0.01982708 0.06657322 0.0228755801
## tra 0.01881445 3.976043e-01 0.14138693 0.07902306 0.2990449299
## log 0.52072524 1.581782e-01 0.14834901 0.04324589 0.0691584460
## edu 0.61910038 1.863327e-02 0.18031698 0.01343043 0.1010061079
## acs 0.87097293 1.009710e-02 0.02760038 0.02272269 0.0108857897
## aco 0.08349875 6.521581e-01 0.14025322 0.04086605 0.0001346848
## def 0.37496348 4.671906e-02 0.06754940 0.40547057 0.0237163328
## det 0.78987471 9.088477e-02 0.02585656 0.03215221 0.0309278329
## div 0.30043759 1.262575e-02 0.28790951 0.25467765 0.0339182311
\end{verbatim}

\begin{Shaded}
\begin{Highlighting}[]
\FunctionTok{fviz\_pca\_var}\NormalTok{(res.pca)}
\end{Highlighting}
\end{Shaded}

\includegraphics{acp_files/figure-latex/unnamed-chunk-7-1.pdf}

\begin{Shaded}
\begin{Highlighting}[]
\FunctionTok{fviz\_pca\_var}\NormalTok{(res.pca, }\AttributeTok{col.var=}\StringTok{"cos2"}\NormalTok{) }\SpecialCharTok{+}
  \FunctionTok{scale\_color\_gradient2}\NormalTok{(}\AttributeTok{low=}\StringTok{"white"}\NormalTok{, }\AttributeTok{mid=}\StringTok{"blue"}\NormalTok{, }
                        \AttributeTok{high=}\StringTok{"red"}\NormalTok{, }\AttributeTok{midpoint=}\FloatTok{0.6}\NormalTok{) }\SpecialCharTok{+} 
  \FunctionTok{theme\_minimal}\NormalTok{()}
\end{Highlighting}
\end{Shaded}

\includegraphics{acp_files/figure-latex/unnamed-chunk-7-2.pdf}

\hypertarget{interpruxe9tation-de-la-carte-des-individus}{%
\section{Interprétation de la carte des
individus}\label{interpruxe9tation-de-la-carte-des-individus}}

\begin{Shaded}
\begin{Highlighting}[]
\NormalTok{ind }\OtherTok{\textless{}{-}} \FunctionTok{get\_pca\_ind}\NormalTok{(res.pca)}
\NormalTok{ind}
\end{Highlighting}
\end{Shaded}

\begin{verbatim}
## Principal Component Analysis Results for individuals
##  ===================================================
##   Name       Description                       
## 1 "$coord"   "Coordinates for the individuals" 
## 2 "$cos2"    "Cos2 for the individuals"        
## 3 "$contrib" "contributions of the individuals"
\end{verbatim}

\begin{Shaded}
\begin{Highlighting}[]
\NormalTok{(ind}\SpecialCharTok{$}\NormalTok{coord)}
\end{Highlighting}
\end{Shaded}

\begin{verbatim}
##           Dim.1       Dim.2       Dim.3       Dim.4       Dim.5
## 1872 -2.9005388  1.02442948  1.56458773  0.48612207 -2.05602528
## 1880 -2.7673894  2.01195321 -0.16951033  1.48359067  1.23613108
## 1890 -2.4163158  0.22401426  0.76571631 -0.26747651 -0.70961060
## 1900 -2.0566342  0.75515473  1.00681672 -0.52267933 -0.50351900
## 1903 -2.3378578  0.16724592  0.62255906  0.18190536  1.21221068
## 1906 -1.9851416  0.62613750  0.69239860 -0.70971498  0.15131836
## 1909 -1.9073550  0.81222235  0.98668748 -0.20099544  0.59499008
## 1912 -1.4310705  0.76841935  0.19417059 -1.29003301  0.78099763
## 1915 -2.1391748  0.95590969 -1.74719977 -1.06270281  0.62618909
## 1923 -1.1429101 -2.88394967 -0.86563257  0.43753507 -0.54236280
## 1926 -1.6740880 -2.61095464  0.49930497  1.76177239 -1.19293695
## 1929 -1.1734318 -1.83119786 -0.61040849  1.15584775  0.69527586
## 1932  0.2706376 -1.95931965 -1.46199017  0.04025195  0.34549190
## 1935  0.6590494 -2.29620943 -0.66272161 -0.30774112  0.82365634
## 1938 -0.4023984 -1.34296396 -0.85017542 -1.84847839  0.06258575
## 1947  1.0812810  2.25116584 -1.27646517 -0.23102315 -0.37302436
## 1950  2.3728087  2.17540030 -1.91746895  2.66338887 -0.18697295
## 1953  1.2037765  1.13432281 -1.66242128 -0.75163234 -0.96195211
## 1956  2.9279665  0.23069051 -0.58985134 -0.44363712 -0.51099614
## 1959  2.6861971  0.14018403  0.07139053 -0.69046798 -1.21198552
## 1962  3.0547171 -0.11078962  0.58640358 -0.64565403 -0.41919951
## 1965  3.1430131  0.31123287  1.41237644  0.76265891  0.93122740
## 1968  3.6956105 -0.46665277  2.29714455  0.28126906  0.47558012
## 1971  3.2392487 -0.08644524  1.11428854 -0.28210591  0.73293093
\end{verbatim}

\begin{Shaded}
\begin{Highlighting}[]
\NormalTok{(ind}\SpecialCharTok{$}\NormalTok{cos2)}
\end{Highlighting}
\end{Shaded}

\begin{verbatim}
##            Dim.1        Dim.2        Dim.3        Dim.4        Dim.5
## 1872 0.465195826 0.0580286677 0.1353562469 0.0130667970 0.2337412510
## 1880 0.451053264 0.2384092199 0.0016923090 0.1296329738 0.0899945790
## 1890 0.785340922 0.0067499753 0.0788654247 0.0096232582 0.0677315541
## 1900 0.604823334 0.0815429281 0.1449490077 0.0390647515 0.0362531819
## 1903 0.572526166 0.0029300177 0.0405994618 0.0034661724 0.1539272013
## 1906 0.710044444 0.0706385838 0.0863803420 0.0907549876 0.0041255874
## 1909 0.601324866 0.1090425407 0.1609182607 0.0066775708 0.0585147763
## 1912 0.406826426 0.1172961265 0.0074895196 0.3305892530 0.1211676031
## 1915 0.439478773 0.0877564611 0.2931774088 0.1084598575 0.0376579234
## 1923 0.106976434 0.6811446166 0.0613664713 0.0156779659 0.0240903859
## 1926 0.192562493 0.4683966370 0.0171295868 0.2132626076 0.0977800139
## 1929 0.182954361 0.4455511712 0.0495071790 0.1775122418 0.0642305066
## 1932 0.008763016 0.4592911458 0.2557209968 0.0001938435 0.0142808225
## 1935 0.060768761 0.7376790488 0.0614478535 0.0132499893 0.0949153711
## 1938 0.024559499 0.2735494427 0.1096286859 0.5182459014 0.0005940978
## 1947 0.122975722 0.5330368206 0.1713800132 0.0056137524 0.0146358054
## 1950 0.250564610 0.2106069308 0.1636254935 0.3156919096 0.0015557932
## 1953 0.183322255 0.1627784003 0.3496275301 0.0714717397 0.1170658822
## 1956 0.734770214 0.0045611967 0.0298197825 0.0168684467 0.0223797159
## 1959 0.754672598 0.0020553222 0.0005330452 0.0498620238 0.1536305853
## 1962 0.888080097 0.0011681760 0.0327267876 0.0396743561 0.0167244441
## 1965 0.727015165 0.0071288913 0.1468087760 0.0428067039 0.0638208491
## 1968 0.680072462 0.0108435255 0.2627601260 0.0039393697 0.0112623804
## 1971 0.646230811 0.0004602367 0.0764705839 0.0049014373 0.0330845453
\end{verbatim}

\begin{Shaded}
\begin{Highlighting}[]
\NormalTok{(ind}\SpecialCharTok{$}\NormalTok{contrib)}
\end{Highlighting}
\end{Shaded}

\begin{verbatim}
##            Dim.1       Dim.2       Dim.3        Dim.4       Dim.5
## 1872  7.04848399  2.13310705  7.90723377  0.991252942 24.86774038
## 1880  6.41621496  8.22781021  0.09281477  9.232556976  8.98893992
## 1890  4.89154157  0.10199992  1.89391522  0.300099378  2.96223571
## 1900  3.54366368  1.15909866  3.27435356  1.145946816  1.49145839
## 1903  4.57904118  0.05685381  1.25194674  0.138798510  8.64441553
## 1906  3.30157636  0.79687085  1.54859231  2.112816360  0.13469847
## 1909  3.04790510  1.34090615  3.14473441  0.169459615  2.08256650
## 1912  1.71577543  1.20017646  0.12178439  6.980643467  3.58821832
## 1915  3.83381277  1.85730078  9.86074808  4.737153545  2.30669628
## 1923  1.09436581 16.90533844  2.42042659  0.803007407  1.73045061
## 1926  2.34798305 13.85629651  0.80529779 13.019473100  8.37171440
## 1929  1.15359693  6.81583485  1.20355481  5.603960575  2.84376514
## 1932  0.06136412  7.80295499  6.90420440  0.006796224  0.70219084
## 1935  0.36389350 10.71695750  1.41868834  0.397250890  3.99090618
## 1938  0.13565967  3.66587141  2.33475774 14.332520208  0.02304254
## 1947  0.97952499 10.30062301  5.26311280  0.223874603  0.81856622
## 1950  4.71697757  9.61893262 11.87630701 29.755206546  0.20565392
## 1953  1.21403196  2.61530201  8.92702840  2.369761928  5.44360147
## 1956  7.18241634  0.10817028  1.12385394  0.825561752  1.53608290
## 1959  6.04524814  0.03994345  0.01646288  1.999773869  8.64120460
## 1962  7.81772434  0.02494863  1.11075419  1.748612341  1.03376294
## 1965  8.27619576  0.19688808  6.44355692  2.439801590  5.10141806
## 1968 11.44223200  0.44262527 17.04517197  0.331846875  1.33053653
## 1971  8.79077077  0.01518905  4.01069898  0.333824483  3.16013414
\end{verbatim}

\begin{Shaded}
\begin{Highlighting}[]
\FunctionTok{fviz\_pca\_ind}\NormalTok{(res.pca,}\AttributeTok{geom =} \StringTok{"point"}\NormalTok{,}\AttributeTok{col.ind.sup =} \StringTok{\textquotesingle{}gray\textquotesingle{}}\NormalTok{)}
\end{Highlighting}
\end{Shaded}

\includegraphics{acp_files/figure-latex/unnamed-chunk-12-1.pdf}

\begin{Shaded}
\begin{Highlighting}[]
\FunctionTok{fviz\_pca\_ind}\NormalTok{(res.pca,}\AttributeTok{geom =} \StringTok{"text"}\NormalTok{,}\AttributeTok{col.ind.sup =} \StringTok{\textquotesingle{}gray\textquotesingle{}}\NormalTok{)}
\end{Highlighting}
\end{Shaded}

\includegraphics{acp_files/figure-latex/unnamed-chunk-12-2.pdf}

\begin{Shaded}
\begin{Highlighting}[]
\FunctionTok{fviz\_pca\_ind}\NormalTok{(res.pca,}\AttributeTok{geom =} \StringTok{"text"}\NormalTok{,}\AttributeTok{col.ind=}\StringTok{"cos2"}\NormalTok{)}\SpecialCharTok{+}
\FunctionTok{scale\_color\_gradient2}\NormalTok{(}\AttributeTok{low=}\StringTok{"blue"}\NormalTok{, }\AttributeTok{mid=}\StringTok{"white"}\NormalTok{, }
                      \AttributeTok{high=}\StringTok{"red"}\NormalTok{, }\AttributeTok{midpoint=}\FloatTok{0.5}\NormalTok{)}
\end{Highlighting}
\end{Shaded}

\includegraphics{acp_files/figure-latex/unnamed-chunk-13-1.pdf}

\begin{Shaded}
\begin{Highlighting}[]
\FunctionTok{plot.PCA}\NormalTok{(res.pca, }\AttributeTok{axes=}\FunctionTok{c}\NormalTok{(}\DecValTok{1}\NormalTok{, }\DecValTok{2}\NormalTok{), }\AttributeTok{choix=}\StringTok{"ind"}\NormalTok{, }\AttributeTok{cex=}\FloatTok{0.7}\NormalTok{)}
\end{Highlighting}
\end{Shaded}

\includegraphics{acp_files/figure-latex/unnamed-chunk-14-1.pdf}

\end{document}
